\documentclass[11pt]{article}\usepackage[]{graphicx}\usepackage[]{color}
%% maxwidth is the original width if it is less than linewidth
%% otherwise use linewidth (to make sure the graphics do not exceed the margin)
\makeatletter
\def\maxwidth{ %
  \ifdim\Gin@nat@width>\linewidth
    \linewidth
  \else
    \Gin@nat@width
  \fi
}
\makeatother

\definecolor{fgcolor}{rgb}{0.345, 0.345, 0.345}
\newcommand{\hlnum}[1]{\textcolor[rgb]{0.686,0.059,0.569}{#1}}%
\newcommand{\hlstr}[1]{\textcolor[rgb]{0.192,0.494,0.8}{#1}}%
\newcommand{\hlcom}[1]{\textcolor[rgb]{0.678,0.584,0.686}{\textit{#1}}}%
\newcommand{\hlopt}[1]{\textcolor[rgb]{0,0,0}{#1}}%
\newcommand{\hlstd}[1]{\textcolor[rgb]{0.345,0.345,0.345}{#1}}%
\newcommand{\hlkwa}[1]{\textcolor[rgb]{0.161,0.373,0.58}{\textbf{#1}}}%
\newcommand{\hlkwb}[1]{\textcolor[rgb]{0.69,0.353,0.396}{#1}}%
\newcommand{\hlkwc}[1]{\textcolor[rgb]{0.333,0.667,0.333}{#1}}%
\newcommand{\hlkwd}[1]{\textcolor[rgb]{0.737,0.353,0.396}{\textbf{#1}}}%
\let\hlipl\hlkwb

\usepackage{framed}
\makeatletter
\newenvironment{kframe}{%
 \def\at@end@of@kframe{}%
 \ifinner\ifhmode%
  \def\at@end@of@kframe{\end{minipage}}%
  \begin{minipage}{\columnwidth}%
 \fi\fi%
 \def\FrameCommand##1{\hskip\@totalleftmargin \hskip-\fboxsep
 \colorbox{shadecolor}{##1}\hskip-\fboxsep
     % There is no \\@totalrightmargin, so:
     \hskip-\linewidth \hskip-\@totalleftmargin \hskip\columnwidth}%
 \MakeFramed {\advance\hsize-\width
   \@totalleftmargin\z@ \linewidth\hsize
   \@setminipage}}%
 {\par\unskip\endMakeFramed%
 \at@end@of@kframe}
\makeatother

\definecolor{shadecolor}{rgb}{.97, .97, .97}
\definecolor{messagecolor}{rgb}{0, 0, 0}
\definecolor{warningcolor}{rgb}{1, 0, 1}
\definecolor{errorcolor}{rgb}{1, 0, 0}
\newenvironment{knitrout}{}{} % an empty environment to be redefined in TeX

\usepackage{alltt}
\usepackage[margin=0.5in]{geometry}
\linespread{1.10}
%\setlength{\parskip}{1ex}
%\setlength{\parindent}{0pt}
% \usepackage{parskip}
\usepackage{siunitx}
\sisetup{group-separator={,},
group-minimum-digits=6,
inter-unit-product=\ensuremath{{}\cdot{}}}
\usepackage{cofi}
\usepackage{fourier}
\usepackage{titling}
% \usepackage[printwatermark]{xwatermark}
% \newwatermark[allpages,color=black!20,angle=-55,scale=6,xpos=30,ypos=10]{DRAFT}
\setlength{\droptitle}{-0.6in}
\pretitle{\Large}
\title{MAT-150}
\posttitle{\hfill}
\preauthor{\hfill\Large}
\author{Project 1: Wildfire Season}
\postauthor{\hfill}
\predate{\hfill\Large}
\date{February 27, 2019}
\postdate{}
% \usepackage{Sweave}
\IfFileExists{upquote.sty}{\usepackage{upquote}}{}
\begin{document}
%\SweaveOpts{concordance=TRUE}

\thispagestyle{empty}
\maketitle
\thispagestyle{empty} \pagestyle{empty}







% \section*{Group ID: id}

\section*{Project goals}
\begin{compactitem}
    \item Practice modeling a real-world phenomenon that is not a routine
        homework exercise
    \item Become more comfortable with open-ended problems
    \item Practice communicating and explaining mathematical results clearly    and professionally
    \item Practice working in a team on a mathematical problem
\end{compactitem}


\subsection*{Project synopsis}
% \vspace{-2.8ex}

    You and your group members are mathematical consultants coordinating with
    several government agencies, including the Bureau of Land Management, the
    Department of the Interior, and the National Interagency Fire Center.
    These agencies are tasked with the development of a model to predict the
    length of the fire season in the a geographic region called the Great
    Basin. Your group is aware that fire prediction and mapping is a highly
    refined science, but also appreciates the value of a good initial
    estimate, obtained without much expense. Part of the team's objective is
    to conduct a sensitivity analysis, since fire season outcomes seem to be
    more variable in recent years. More is explained below.

    Your team had an opportunity to consult with more experienced
    fire scientists and you discussed some of the basic ideas of wildfire
    prediction. They even gave you some formulas, but the notes from the
    meeting were lost, and you weren't too sure you understood all of it
    anyway. However, you do remember that they said that you can get
    modestly good estimates using dimensional analysis. More details are
    included below.

\subsection*{Useful facts and objectives}
% \vspace{-2.8ex}

    \begin{compactitem}

        \item Unlike many scenarios you have trained on, this situation
        requires a proportionality constant that carries dimensions: namely,
        $[K] = L^{1} M^{1} \Theta^{-1} T^{-1.50}$.
        You hope to estimate the constant $K$ from the data you have.

        \item Lacking a more sophisticated model, your team agrees that
        dimensional analysis might help in determining an expression for the
        length of the fire season, $S$, in terms of the May--June average
        maximum daily temperature, $Q$, and some other important quantities,
        namely:
        \begin{compactenum}
            \item the January--June average accumulated precipitation, $P$;
            \item the mass density of vegetation per unit area, $\delta$; and
            \item the May--June average cumulative temperature excess, $X$. This
            is measured in \emph{degree-days} \si{\celsius.day}.
        \end{compactenum}

        \item Data indicate that the May--June average maximum daily temperature in
        the Great Basin in 2017 was
        \SI{18.78}{\celsius}. The January--June average
        accumulated precipitation in 2017 was
        \SI{54.75}{mm}. The density of vegetation is
        estimated to be about $\SI{6.61}{kg/m^2}$ in this region.
        The May--June average cumulative temperature excess for 2017 was
        $\SI{22.86}{\celsius.day}$. Finally, the 2017 fire season
        lasted $\num{158}$ days. You will base your model on these
        data.

        \item Weather patterns have been more highly variable in the past several years
        compared to historical trends. Consider the effects that slight changes to the
        quantities in your model will have on the length of the fire season. For the
        each change that you consider, compute the length of the fire season and give the
        percent change in season length induced by each of your changes. (Note: a contour
        plot is a useful tool to see the values of a function of two variables.)

    \end{compactitem}

% \newpage

    Your team must now pull together all of its ideas to calculate an estimate
    of the Great Basin fire season in 2019 under several possible conditions. You
    should report all answers with three significant digits. Once you finish
    your calculations, your team should write a professional report restating
    the problem and summarizing your work and your findings. Your report will be
    read by the official who determine the fire management budget and strategy, so
    a rough report just won't do. It must clearly explain your calculations and
    results. Further, your report must contain at least one
    meaningful, carefully labeled graph that adds something to the argument
    your team is making.

 %\newpage

\subsection*{Timeline of project}
% \vspace{-2.8ex}

\begin{compactdesc}
    \item[Preliminary check-in:] Thursday/Friday, March 7/8, with your
    instructor at a scheduled time.
    \item[Final report due:] Wednesday, March 13, at the beginning of
    class, in Canvas.
    \item[Group evaluations due:] Wednesday, March 13, by 11:59pm, in
    Canvas.
    \item[Final interview:] Thursday/Friday, March 14/15, with your
    instructor at a scheduled time.
    \item[Late papers/group evaluations:] will not be accepted.
\end{compactdesc}

\subsection*{Preliminary check-in and final interview meetings}
% \vspace{-2.8ex}

Your group turns in one report in Canvas and everyone in the group gets the
same grade for the Written Report section for this project. The Written Report
accounts for $35\%$ of your Project 1 grade.  The rest of your score ($65\%$)
is earned during the Preliminary Check-in and Final Interview meetings with
your instructor. In these meetings, you will answer a question regarding the
development of the report's results to demonstrate your participation in the
project. Your group will also submit the introduction to your report in Canvas
(see grading rubric). You are expected to participate in the project,
understanding both the solution to the problem and the process by which it was
obtained.

\subsection*{Written report}
% \vspace{-2.8ex}

Use Microsoft Word or another suitable word processor. Type all equations,
using (for example) the Microsoft Equation Editor built into Word and standard
mathematical notation. Use RStudio to make all graphs. Graphs must include
proper axis labels, captions/titles, etc. Group members' names must be clearly
visible on the front page of your report. Margins will not exceed 1 inch and
lines are to be no more than 1.5-spaced. Use a font such as Palatino or Times
New Roman with a font size of no more than 12 points.

Reports will be self-contained, i.e., the reader will not need prior knowledge
or understanding of the problem to understand your report. Your calculations
and conclusions will be explained in detail, as befits a professional report.
This report will be read by people of varying mathematical background, so do
your best to keep your report at a level accessible to the widest possible
audience (in other words, try to write for the general public whenever
possible). Your explanation and presentation are as important as the
mathematics, but of course your mathematical analysis must be clear, complete,
and correct. It should go without saying that your grammar, punctuation, and
spelling will be flawless. All outside sources, including websites, used in
your report must be properly cited.

\subsection*{Group evaluation}
% \vspace{-2.8ex}

Every group member is required to submit an informal group valuation
\emph{before midnight on the same day the reports are due} in Canvas. The
evaluation is at most a short paragraph about how well the group worked
together. The evaluation \emph{must} include your estimate of the share of the
project work that you feel each group member did, expressed as a percentage
(e.g.~33\%/33\%/33\% or 40\%/30\%/30\%, etc.). These percentages must add to
100\%---or 99\% is close enough. These evaluations will be held in confidence,
and used at the end of the semester, if necessary, to adjust final grades.

\subsection*{A word to the wise}
% \vspace{-2.8ex}

Make sure that your report's qualities of clarity, completeness, and
correctness reflect your best abilities. Carefully read all the instructions
and resources provided.

%%%%%%%%%%%% BEGIN ANSWER KEY SECTION %%%%%%%%%%%%%%

\newpage

\section*{Key for group }



Formula: $S = \ensuremath{5.586122\times 10^{6}} Q^{-1.5} P^{-3} \delta^{-1} X^{2.5}$

Formula if convert to meters: $S = \ensuremath{5.586122\times 10^{15}} Q^{-1.5} P^{-3} \delta^{-1} X^{2.5}$

%%%%%%%%%%%% END ANSWER KEY SECTION %%%%%%%%%%%%%%

\end{document}
